\documentclass{article}
\title{Antisèche de Physique}
\author{Anya Voeffray \thanks{thanks to no fucking one, I hate Physics}}
\date{Septembre 2024 - Juillet 2026}
\usepackage{gensymb}

\begin{document}

\begin{titlepage}
\maketitle
\begin{equation}
	Mes Capacites En Physique =  \frac{Motivation \cdot Capacites En Maths}{Annee Depuis Le Dernier Cours De Physique}
\end{equation}


\end{titlepage}

\section{Masse Volumique ($\varphi$)}

La masse volumique permet de définir la masse d'une matière pour $1m^3$ de cette même matière.

Par exemple ici, la masse volumique de l'eau:
\begin{equation}
	\varphi = 1000 \frac{kg}{m^3}
\end{equation}

Dans un exemple pratique, si on se retrouve avec $3l$ d'eau, pour calculer sa masse, cela donnerait:
	\begin{equation}
		3l = 3dm^3 = 0.003m^3
	\end{equation}
	
	\begin{equation}
		1000 \cdot 0.003 = 3kg
	\end{equation}

En résumé, voici les trois équations à se remémorer:
\begin{equation}
	\varphi = \frac{m}{V}
\end{equation}

\begin{equation}
	m = \varphi \cdot V
\end{equation}

\begin{equation}
	V = \frac{m}{\varphi}
\end{equation}

\pagebreak

\section{Chaleur Massique ($C$)}

La chaleur massique représente la quantité de Joules nécessaires pour augmenter la température de $1kg$ d'une matière donnée de $1 ^\circ C$

\begin{equation}
	C = \frac{J}{kg \cdot \theta}
\end{equation}

Sachant que la chaleur massique de l'eau est comme suit: $1480 \frac{J}{kg \cdot \theta}$

Il est donc possible de calculer l'augmentation de la température ($\theta$) de $126kg$ d'eau à $20^\circ C$ si on lui applique $30000J$:

\begin{equation}
	\theta = \frac{J}{kg \cdot C}
\end{equation}

Une fois les données entrées:

\begin{equation}
	1.61\mathrm{e}{-1} = \frac{30000}{126 \cdot 1480}
\end{equation}
Dans l'ensemble, ajouter $30000 joules$ à $126kg$ d'eau résulte
en une augmentation de $1.61\mathrm{e}{-1}$ degrés celsius.

\subsection{Transfert d'énergie}

Si on veut calculer la quantité d'énergie nécessaire pour passer
d'une température données à une autre, il faut suivre l'équation
suivante:

\begin{equation}
	Q = m \cdot C \cdot \Delta \theta
\end{equation}

$\Delta\theta$ représente la différence de température calculée comme suis:
\begin{equation}
	\theta_{final} - \theta_{initiale}
\end{equation}

Exemple:
J'aimerais chauffer $10kg$ d'eau depuis $20^\circ C$ vers $100^\circ C$:
\begin{equation}
	Q = m_{eau} \cdot C_{eau} \cdot (100 - 20)
\end{equation}

Une fois avec toutes les informations:
\begin{equation}
	3.344\mathrm{e}{6} = 10 \cdot 4180 \cdot 80
\end{equation}
Il faudra donc dépenser $3.344\mathrm{e}{6}$ Joules pour chauffer $10kg$ d'eau de 80 degrés.

\subsection{Transfert avec plusieurs matériaux}

Pour calculer le transfert d'énergie entre plusieurs matériaux dans un environnement sans perte, il faudra suivre l'équation suivante.

\begin{equation}
	0 = m_1 \cdot C_1 \cdot \Delta\theta_1 + m_2 \cdot C_2 \cdot \Delta\theta_2
	% 0 = m_1 \cdot C_1 \cdot (\theta_{finale} - \theta_{dep1}) + m_2 \cdot C_2 \cdot (\theta_{finale} - \theta_{dep2})
\end{equation}

Il arrivera parfois de manquer d'une information. Voici les équations avec les différents éléments isolés

La température finale:
\begin{equation}
	\theta_{finale} = \frac{m_1 \cdot C_1 \cdot \theta_{dep1} + m_2 \cdot C_2 \cdot \theta_{dep2}}{m_1 \cdot C_1 + m_2 \cdot C_2}
\end{equation}

Une des chaleurs massiques:
\begin{equation}
	C_1 = \frac{m_2 \cdot C_2 \cdot \Delta\theta_2}{m_1 \cdot \Delta\theta_1} \cdot -1
\end{equation}

Une des températures de départ:
\begin{equation}
	\theta_{dep1} = (\frac{m_2 \cdot C_2 \cdot \Delta\theta_2}{m_1 \cdot C_1} - \theta_{finale}) \cdot -1
\end{equation}

Une des masses:
\begin{equation}
	m_1 = \frac{m_2 \cdot C_2 \cdot \Delta\theta_2}{C_1 \cdot \Delta\theta_1} \cdot -1
\end{equation}

Il sera aussi possible que la chaleur massique de la matière ne soit pas existante. C'est à dire que c'est un agrégat de plusieurs matières, comme un thermos ou une cafetière. Dans ce cas, il faudra remplacer le $m \cdot C$ de la matière par son $\mu$.

Le $\mu$ sera obligatoirement donné, à moins qu'il soit l'inconnu de l'équation. Dans le premier cas, voici ce que cela change à l'équation pour deux matières différentes.

\begin{equation}
	0 = m_1 \cdot C_1 \cdot \Delta\theta_1 + \mu_2 \cdot \Delta\theta_2
\end{equation}

Dans le deuxième cas, voici l'équation qu'il faudra poser pour trouver le $\mu$ de la matière:

\begin{equation}
	\mu_2 = \frac{m_1 \cdot C_1 \cdot \Delta\theta_1}{\Delta\theta_2} \cdot -1
\end{equation}

\subsection{Transfert d'énergie avec changement d'état de la matière}

Lors d'un transfert d'énergie, il est possible que la matière change d'état, 
par exemple de la glace qui va passer de $-10\celsius$ à $10\celsius$, 
elle va changer d'état et passer de solide à liquide, en l'occurrence de glace à eau.
\newline
C'est cette étape qui va consommer le plus d'énergie. La quantité d'énergie nécessaire pour
passer de solide à liquide est propre à chaque matière et s'appelle
la "Chaleur latente de fusion" ($L_f$). On la retrouve dans des tableau, elle est fixe.
\newline
\newline
Par exemple on retrouvera ces valeurs:
\newline
\newline
Glace (Dont l'état liquide est l'eau) $\rightarrow 3.3\cdot10^5$
\newline
Aluminium $\rightarrow 3.96 \cdot 10^5$
\newline
\newline

On va donc retrouver une équation légèrement modifiée pour calculer l'énergie
nécessaire au réchauffement de ladite matière:

\begin{equation}
	Q = m_1 \cdot C_1solide \cdot (\theta_{fusion} - \theta_{dep}) + L_f1 \cdot m_1 + m_1 \cdot C_1liquide \cdot (\theta_{final} - \theta_{solidification})
\end{equation}

On se retrouve dans la logique suivante:
\newline
\newline
Il faut d'abord mener le solide jusqu'à $\theta_{fusion}$ puis y ajouter la Chaleur
latente de fusion ($L_f1 \cdot m_1$) et finalement ajouter l'équation pour passer
le liquide à sa température finale ($m_1 \cdot C_1liquide \cdot (\theta_{final} - \theta_{solidification})$)
\newline
\newline

On se retrouve donc à devoir calculer 3 quantités d'énergie différentes qui vont finalement être additionnées pour connaitre l'énergie nécessaire pour chauffer un solide passé son point de fusion.

\pagebreak
\section{Dilatation}

La dilatation se fait de manière différente pour les solides, liquides et gaz.

\subsection{Solides}

Dans le cas des solides, il faut prendre en compte le nombre de dimensions que vont être prises en compte. On retrouvera plusieurs points:

Dans le cadre des solides, on retrouve des données concernant les matières. Celles-ci sont donc disponibles dans des tableaux fournis par les enseignants:

$\alpha$ : Coefficient de dilatation linéaire $\rightarrow$ ([$\frac{1}{\celsius}$])
\newline $\alpha$ est l'allongement d'une matière (solide) donnée d'une longueur de 1 mètre si on l'augmente de 1$\celsius$
\newline Par exemple: Si on a un file d'acier de 1m de long et qu'on l'augmente de 1$\celsius$, la longueur augmentera de 11$\cdot$$10^{-6}$ mètres 
  
\subsubsection{Linéaire L}

Si on considère que la dilatation est linéaire, on ne calculera que dans une
dimension. Ce que ça implique, c'est que la largeur et la profondeur de l'objet
en question n'est pas assez grande pour influencer quoi qu'il soit dans 
la situation donnée.

\begin{equation}
  L_2 = L_1 + \Delta_L
\end{equation}

\begin{equation}
  L_2 = L_1 + L_1 \cdot \Delta\theta \cdot \alpha
\end{equation}

\begin{equation}
  L_2 = L_1 \cdot (1 + \alpha \cdot \Delta\theta)
\end{equation}

\subsubsection{Surface S}

Si on considère que la dilatation est une dilatation de surface, on ne 
calculara que dans 2 dimensions. Cela implique que la profondeur de l'objet ne 
sera pas prise en compte car on considère qu'elle n'est pas assez élevée pour 
influencer la situation donnée.

\begin{equation}
  S_2 = S_1 \cdot (1 + 2 \cdot \alpha \cdot \Delta\theta)
\end{equation}

\subsubsection{Volume V}

Si on considère que la dilatation est une dilatation volumique, on calculera la dilatation dans les 3 dimensions, longueur, largeur, profondeur.

\begin{equation}
  V_2 = V_1 \cdot (1 + 3 \cdot \alpha \cdot \Delta\theta)
\end{equation}

\subsection{Liquides}

Les liquides ne sont que calculés avec le volume. On y retrouve donc qu'une 
seule équation.
\newline $\gamma$, dans ce cas remplace $\alpha$. C'est le coefficient de 
dilatation volumique.
\newline A noter aussi que le coefficient de dilatation des liquides est 1'000 
fois plus grand que celui des solides, c'est pour ça qu'il n'est pas possible 
de calculer la dilatation d'un liquide dans la surface ou la linéarité.

\begin{equation}
  V_2 = V_1 \cdot (1 + \gamma \cdot \Delta\theta)
\end{equation}

\pagebreak

\section{Loi des gaz}

\subsection{Grandeurs physique}

\begin{itemize}
  \item Volume [$m^3$]
  \item Pression [bar] / [Pa]
  \item Température [K] $\rightarrow$ T \newline Pour passer de $\celsius$ à Kelvin il faut ajouter 273,15 aux Celsius
  \item Pression atmosphérique $\rightarrow$ 1 [bar] / 100'000 [Pa]
\end{itemize}

\subsection{Conditions}

\begin{itemize}
  \item La Température [T] doit être en valeur absolue, donc en Kelvin
  \item La pression doit être en valeur absolue
  \item La quantité de gaz ne doit pas changer
\end{itemize}

\subsection{Equations}

\begin{equation}
  \frac{p_0 \cdot V_0}{T_0} = \frac{p_1 \cdot V_1}{T_1}
\end{equation}

\begin{equation}
  p_0 \cdot V_0 = p_1 \cdot V_1
\end{equation}

\begin{equation}
  \frac{V_0}{T_0} = \frac{V_1}{T_1}
\end{equation}

\begin{equation}
  \frac{p_0}{T_0} = \frac{p_1}{T_1}
\end{equation}

\pagebreak
\section{Electricité}

\pagebreak
\section{Cinétique}

La cinétique est la description des mouvements d'un corps dans l'espace sans faire
intervenir la les forces génératrices de ces mouvements.

\subsection{MRU (Mouvement Rectiligne Uniforme)}

Le MRU est utilisé si le mouvement n'accélère ni ne décelère. Dans l'équation on retrouve:
\begin{itemize}
	\item X $\rightarrow$ Position de l'objet par rapport au référentiel
	\item V $\rightarrow$ La vitesse de déplacement
	\item t $\rightarrow$ Le temps écoulé entre le début et la fin de la mesure en seconde
\end{itemize}

Voici donc l'équation à effectuer:

\begin{equation}
  X + V \cdot t
\end{equation}

\subsection{MRUA (Mouvement Rectiligne Uniformément Acceléré)}

Le MRUA, contrairement au MRU, est accéléré de manière uniforme, il n'y a donc pas de changement d'accélération, mais la vitesse augmente avec le temps.

Dans l'équation, on retrouve:
\begin{itemize}
	\item a $\rightarrow$ L'accélération
	\item t $\rightarrow$ Le temps écoulé entre le début et la fin de la mesure en seconde
	\item X $\rightarrow$ Position de l'objet par rapport au référentiel
	\item V $\rightarrow$ La vitesse de déplacement au départ (t = 0)
\end{itemize}

Voici donc l'équation:

\begin{equation}
	\frac{1}{2} \cdot a \cdot t^2 + X + V * t
\end{equation}



\end{document}
