\documentclass{article}
\title{Antisèche de Physique}
\author{Anya Voeffray \thanks{thanks to no fucking one, I hate Physics}}
\date{Octobre 2024}

\begin{document}

\begin{titlepage}
\maketitle
\begin{equation}
	Mes Capacites En Physique =  \frac{Motivation \cdot Capacites En Maths}{Annee Depuis Le Dernier Cours De Physique}
\end{equation}


\end{titlepage}

\section{Masse Volumique ($\varphi$)}

La masse volumique permet de définir la masse d'une matière pour $1m^3$ de cette même matière.

Par exemple ici, la masse volumique de l'eau:
\begin{equation}
	\varphi = 1000 \frac{kg}{m^3}
\end{equation}

Dans un exemple pratique, si on se retrouve avec $3l$ d'eau, pour calculer sa masse, cela donnerait:
	\begin{equation}
		3l = 3dm^3 = 0.003m^3
	\end{equation}
	
	\begin{equation}
		1000 \cdot 0.003 = 3kg
	\end{equation}

En résumé, voici les trois équations à se remémorer:
\begin{equation}
	\varphi = \frac{m}{V}
\end{equation}

\begin{equation}
	m = \varphi \cdot V
\end{equation}

\begin{equation}
	V = \frac{m}{\varphi}
\end{equation}

\pagebreak

\section{Chaleur Massique ($C$)}

La chaleur massique représente la quantité de Joules nécessaires pour augmenter la température de $1kg$ d'une matière donnée de $1 ^\circ C$

\begin{equation}
	C = \frac{J}{kg \cdot \theta}
\end{equation}

Sachant que la chaleur massique de l'eau est comme suit: $1480 \frac{J}{kg \cdot \theta}$

Il est donc possible de calculer l'augmentation de la température ($\theta$) de $126kg$ d'eau à $20^\circ C$ si on lui applique $30000J$:

\begin{equation}
	\theta = \frac{J}{kg \cdot C}
\end{equation}

Une fois les données entrées:

\begin{equation}
	1.61\mathrm{e}{-1} = \frac{30000}{126 \cdot 1480}
\end{equation}
Dans l'ensemble, ajouter $30000 joules$ à $126kg$ d'eau résulte en une augmentation de $1.61\mathrm{e}{-1}$ degrés celsius.

\subsection{Transfert d'énergie}

Si on veut calculer la quantité d'énergie nécessaire pour passer d'une température données à une autre, il faut suivre l'équation suivante:

\begin{equation}
	Q = m \cdot C \cdot \Delta \theta
\end{equation}

$\Delta\theta$ représente la différence de température calculée comme suis:
\begin{equation}
	\theta_{final} - \theta_{initiale}
\end{equation}

Exemple:
J'aimerais chauffer $10kg$ d'eau depuis $20^\circ C$ vers $100^\circ C$:
\begin{equation}
	Q = m_{eau} \cdot C_{eau} \cdot (100 - 20)
\end{equation}

Une fois avec toutes les informations:
\begin{equation}
	3.344\mathrm{e}{6} = 10 \cdot 4180 \cdot 80
\end{equation}
Il faudra donc dépenser $3.344\mathrm{e}{6}$ Joules pour chauffer $10kg$ d'eau de 80 degrés.

\subsection{Transfert avec plusieurs matériaux}

Pour calculer le transfert d'énergie entre plusieurs matériaux dans un environnement sans perte, il faudra suivre l'équation suivante.

\begin{equation}
	0 = m_1 \cdot C_1 \cdot \Delta\theta_1 + m_2 \cdot C_2 \cdot \Delta\theta_2
	% 0 = m_1 \cdot C_1 \cdot (\theta_{finale} - \theta_{dep1}) + m_2 \cdot C_2 \cdot (\theta_{finale} - \theta_{dep2})
\end{equation}

Il arrivera parfois de manquer d'une information. Voici les équations avec les différents éléments isolés

La température finale:
\begin{equation}
	\theta_{finale} = \frac{m_1 \cdot C_1 \cdot \theta_{dep1} + m_2 \cdot C_2 \cdot \theta_{dep2}}{m_1 \cdot C_1 + m_2 \cdot C_2}
\end{equation}

Une des chaleurs massiques:
\begin{equation}
	C_1 = \frac{m_2 \cdot C_2 \cdot \Delta\theta_2}{m_1 \cdot \Delta\theta_1} \cdot -1
\end{equation}

Une des températures de départ:
\begin{equation}
	\theta_{dep1} = (\frac{m_2 \cdot C_2 \cdot \Delta\theta_2}{m_1 \cdot C_1} - \theta_{finale}) \cdot -1
\end{equation}

Une des masses:
\begin{equation}
	m_1 = \frac{m_2 \cdot C_2 \cdot \Delta\theta_2}{C_1 \cdot \Delta\theta_1} \cdot -1
\end{equation}

Il sera aussi possible que la chaleur massique de la matière ne soit pas existante. C'est à dire que c'est un agrégat de plusieurs matières, comme un thermos ou une cafetière. Dans ce cas, il faudra remplacer le $m \cdot C$ de la matière par son $\mu$.

Le $\mu$ sera obligatoirement donné, à moins qu'il soit l'inconnu de l'équation. Dans le premier cas, voici ce que cela change à l'équation pour deux matières différentes.

\begin{equation}
	0 = m_1 \cdot C_1 \cdot \Delta\theta_1 + \mu_2 \cdot \Delta\theta_2
\end{equation}

Dans le deuxième cas, voici l'équation qu'il faudra poser pour trouver le $\mu$ de la matière:

\begin{equation}
	\mu_2 = \frac{m_1 \cdot C_1 \cdot \Delta\theta_1}{\Delta\theta_2} \cdot -1
\end{equation}


\end{document}